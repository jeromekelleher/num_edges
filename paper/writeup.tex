\documentclass{article}
\usepackage{fullpage}
\usepackage{amsmath}
\usepackage{amssymb}
\usepackage{natbib}

\newcommand{\E}{\mathbb{E}}
\renewcommand{\P}{\mathbb{P}}


\begin{document}


% Name ideas:
% 
% extend edges
% bundle edges
% inflate edges
% bundle lines of descent
% longer ancestral haplotypes
% inflated ancestors
% compress paths
% optimizing edge tables
% reduce number of ancestors
% reduce ancestral paths

\section{Introduction}

% PETER
% * what's a tree sequence
% * why is a tree sequence (motivation)

The pedigree that describes how everyone of some species
are related to each other
is a large graph, where nodes are indexed by time.
Within this, one can trace back the paths along which segement of genome
have been inherited.
The \emph{succinct tree sequence} (or, tree sequence for short)
was introduced by \citet{kelleher2016efficient} to describe the subset of these paths
along which the genome of a sample of individuals was inherited.


\section{Motivation and statement of problem}

% PETER
% * minimize number of edges
% * gives extra info about shared haplotypes, reduces number of ancestral paths


\section{Algorithm}

% * description (HALLEY)
% * proof of something:
%     - arrives at a local minimum?
%     - guess at typical reduction?


\section{Results (how it works)}

% (ADAM)
% * reduction in edges ~ sequence length
% * speed increase ~ sequence length
% * proprotion of added edges that are true ~ sequeence length
% * apply to real data:
%     - % fewer edges
%     - % fewer distinct ancestors (ancestral paths?) of sample at some time


\bibliography{references}

\end{document}
